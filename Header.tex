\documentclass[12pt,a4paper]{report}
\usepackage[utf8x]{inputenc}
\usepackage[T1]{fontenc}
\usepackage{geometry}
\usepackage{graphicx}
\usepackage[english]{isodate}
\usepackage[english]{babel}
\usepackage{url}
\usepackage{color}
\usepackage{ulem}
\usepackage{titlesec}
\usepackage{hyperref}
\usepackage{float}
\usepackage{tabularx}
\usepackage{ucs}
\usepackage[english]{algorithm2e}

% lstlisting to show code nicely
\usepackage{listings}

\definecolor{mygreen}{rgb}{0,0.6,0}
\definecolor{mygray}{rgb}{0.5,0.5,0.5}
\definecolor{mymauve}{rgb}{0.58,0,0.82}

\lstset{
  backgroundcolor=\color{white},   % choose the background color; you 
  %basicstyle=\footnotesize,
  basicstyle=\ttfamily,            % the size of the fonts that are used 
  breakatwhitespace=false,         % sets if automatic breaks should 
  breaklines=true,                 % sets automatic line breaking
  captionpos=b,                    % sets the caption-position to bottom
  commentstyle=\color{mygreen},    % comment style
  deletekeywords={...},            % if you want to delete keywords from 
  escapeinside={\%*}{*)},          % if you want to add LaTeX within 
  inputencoding=utf8x,
  extendedchars=true,              % lets you use non-ASCII characters; 
  frame=single,                    % adds a frame around the code
  keepspaces=true,                 % keeps spaces in text, useful for 
  keywordstyle=\color{blue},       % keyword style
  language={[Sharp]C},             % the language of the code
  morekeywords={*,...},            % if you want to add more keywords to 
  numbers=left,                    % where to put the line-numbers; 
  numbersep=5pt,                   % how far the line-numbers are from 
  numberstyle=\tiny\color{mygray}, % the style that is used for the 
  rulecolor=\color{black},         % if not set, the frame-color may be 
  showspaces=false,                % show spaces everywhere adding 
  showstringspaces=false,          % underline spaces within strings 
  showtabs=false,                  % show tabs within strings adding 
  stepnumber=2,                    % the step between two line-numbers. 
  stringstyle=\color{mymauve},     % string literal style
  tabsize=2,                       % sets default tabsize to 2 spaces
  title=\lstname,                  % show the filename of files included 
  literate={æ}{{\ae}}1
           {ø}{{\o}}1
           {å}{{\aa}}1
           {Æ}{{\AE}}1
           {Ø}{{\O}}1
           {Å}{{\AA}}1
}



% Fonts
\usepackage{helvet} % <-- Helvetica (sans)
\usepackage[sc]{mathpazo} % <-- Palatino (serif)
\linespread{1.05}   % Extra linespread for Palatino

% Extra settings
% \setlength\parindent{0pt} % For at fjerne automatiske indrykninger
% \let\clearpage\relax % For at fjerne alle clearpage sideskift

% Meta (til titelblad)
% HUSK; at rette Jonas i titelbladet.
\newcommand{\rtitle}{Media Recommendation} %<----
\newcommand{\rtheme}{Programmering og problemløsning}
\newcommand{\rperiod}{P3, 2013}
\newcommand{\rdeadline}{20. Dec 2013} %<----Erstattes med rapportens deadline
\newcommand{\rprints}{10} %<---- Erstattes med oplægsantal
\newcommand{\rappendices}{2 + 1 CD} %<---- Erstattes med bilagsantal og art
\newcommand{\rlastpage}{60} %<---- Erstattes med sideantal

%\newcommand{\code}[1]{\lstinline!#1!}
\newcommand{\code}[1]{\lstinline{#1}} 

\usepackage{lastpage}

%relinput
\usepackage{xstring}
\usepackage{etoolbox}

%Init a new array
% #1 = array name
\newcommand\arrayinit[1]{%
	\newcounter{#1-cnt}%
}

%set a value in an array
% #1 = array name
% #2 = array index
% #3 = new value
\newcommand\arraysetvalue[3]{%
	\cslet{#1-#2}{#3}%
}

%append new value to the end af array - push
% #1 = array name
% #2 = append value
\newcommand\arraypush[2]{%
	\stepcounter{#1-cnt}%
	\arraysetvalue{#1}{\arabic{#1-cnt}}{#2}
}

%Get value at position
% #1 = array name
% #2 = array index
% NB: To use counters, do: \arraygetvalue{ARR}{\arabic{COUNTER}}
\newcommand\arraygetvalue[2]{%
	\csuse{#1-#2}%
}

%Delete value at position
% #1 = array name
% #2 = array index
\newcommand\arraydeletevalue[2]{%
	\csundef{#1-#2}%
}

%Delete the last index
% #1 = array name
\newcommand\arraydeletelast[1]{%
	\arraydeletevalue{#1}{\arabic{#1-cnt}}%
	\addtocounter{#1-cnt}{-1}%
}

%Get last value in array and delete the last index; pop
% #1 = array name
\newcommand\arraypop[1]{%
	\arraygetvalue{#1}{\arabic{#1-cnt}}%
	\arraydeletelast{#1}%
}

%\newcounter{tempcount} % for temporary usage

\newcommand\arrayloop[1]{%
\setcounter{tempcount}{0}%
	Looping \arabic{tempcount} to \arabic{#1-cnt}%
	
	\loop\ifnum\value{tempcount}<\value{#1-cnt}%
		\stepcounter{tempcount}%
		\arraygetvalue{#1}{\arabic{tempcount}} \\ %
	\repeat%
}
 %Enable this for debug..
\arrayinit{relinputlayerfilepaths}


\newcommand\lastsubstrposition{}
\newcommand\lastindex[2]{%
	\StrCount{#1}{#2}[\numberofsubstr]%
	\StrPosition[\numberofsubstr]{#1}{#2}[\lastsubstrposition]%
}


\newcommand\relinputfilepath{}
\newcommand\relinputfilename{}
\newcommand\relinputtempstrlen{}
\newcommand\relinputparseinput[1]{%
	%get the pos of the last slash
	\lastindex{#1}{/}%
	
	%Whatever's left of the last slash is the path
	\StrLeft{#1}{\lastsubstrposition}[\relinputfilepath]%
	
	%Whatever's right of the last slash is the filename
	\StrLen{#1}[\relinputtempstrlen]%
	\StrRight{#1}{\the\numexpr\relinputtempstrlen-\lastsubstrposition}[\relinputfilename]%
}


\newcommand\filetoinclude{}
\newcommand\relinput[1]{
	%Parse the input
	\relinputparseinput{#1}%
	
	%Push the path to the array
	\arraypush{relinputlayerfilepaths}{\relinputfilepath}%
	
	%Create the total path to the file to be included
	\relinputinputconcatpath{relinputlayerfilepaths}{\relinputfilename}%
	\renewcommand\filetoinclude{\completepath}%
	\input{\filetoinclude}%
	
	%Delete the last index in the array
	\arraydeletelast{relinputlayerfilepaths}%
}


\global\def\completepath{}
\newcounter{concatpathcounter}
\newcommand\relinputinputconcatpath[2]{
	\global\edef\completepath{}%
	\setcounter{concatpathcounter}{0}%
	\loop\ifnum\value{concatpathcounter}<\value{#1-cnt}%
		\stepcounter{concatpathcounter}%
		\global\edef\completepath{\completepath\arraygetvalue{#1}{\arabic{concatpathcounter}}}%
	\repeat%
	\global\edef\completepath{\completepath#2}%
}


%Forside stuff
\definecolor{Sapphire}{RGB}{28,117,188}
\usepackage[x11names,rgb,usenames,dvipsnames,svgnames,table]{xcolor}
