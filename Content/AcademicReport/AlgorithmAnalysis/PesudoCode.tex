Before the development of the recommendations algorithm, several pseudo code examples was created, to try and formalize how the collaborative side the algorithm should work, implementing the theory described in Section \ref{CollaborativeDes}. In Algorithm \ref{IndexUsers} and \ref{CompareUserPair} you can see the two methods which does the bulk of the collaborative work.

\begin{algorithm}[H]
	\DontPrintSemicolon
	\;
	\textbf{IndexUsers( User U, Set of users A )}\;
	\KwData{User U, and all other users Set A}
	\KwResult{a dictionary D of Set A with a coefficient}
	Initialization:\;
	double k\;
	Dictionary( User, double ) D\;
	\ForEach{User in A}{
		\If{User from A != U}{
			Compare the user pair\;
			k <- CompareUserPair( U, user from A )\;
			D <- Add ( User from A, k ) to the dictionary\;
		}
	}
	Return D\;
	\label{IndexUsers}
	\caption{The IndexUsers method}
\end{algorithm}


\begin{algorithm}[H]
	\DontPrintSemicolon
	\;
	\textbf{CompareUserPair( User U, User A )}\;
	\KwData{Main user U, and user A}
	\KwResult{a coefficient for how similar user U and A are}
	Initialization:\;
	Set X of integers\;
	Set Y of integers\;
	\ForEach{Media in user U's medialist}{
		\If{User A also have Media in their medialist}{
			X <- Add U's rating of this media\;
			Y <- Add A's rating of this media\;
		}
	}
	Apply the Pearson measure to the two sets
	Return Pearson($\sum{X}$, $\sum{Y}$, $\sum{XY}$, $\sum{X^2}$, $\sum{Y^2}$, Size of X)
	\label{CompareUserPair}
	\caption{The CompareUserPair method}
\end{algorithm}

The content-based side of the algorithm works in a similar way, with the difference being the type of data which it works on. The collaborative part compares a users watched media with the watched media this user has in common with every other user in the system. The content-based part compares a vector representation of the user with every vector representations of media, which is of similar length.