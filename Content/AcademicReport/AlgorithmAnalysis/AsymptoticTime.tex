The running time of the algorithm can be analysed using asymptotic notation that defines both the worst and average-case running time of the algorithm. The running time of the loops in Algorithms \ref{IndexUsers} \ref{CompareUserPair} can be erratic, as it depends on how much media a specific user have added to his medialist. The if-structure within the second loop can also be erratic making it difficult to calculate the average-case running time.

The running time is described using Big-O notation, with expressions indicating variables inside the algorithm. The Big-O notation uses the following symbols:
\begin{itemize}
	\item $O(n)$, an upper bound, indicating the highest running time possible.
	\item $\Omega(n)$, a lower bound, indicating the lowest running time possible.
	\item $\Theta(n)$, a tight bound, indicating both of the previous two running-times.
\end{itemize}

The worst-case running time is relatively easy to calculate. The worst-case occurs if every user have the exact same media indicated on their media lists. The worst-case running time is then all the users in the system, except the user in question, times the subset of media available from all the users media in their media lists. The upper bound of the algorithm can then be defined as:

\[
m(n-1) = mn - m = O(mn)
\]

Where $n$ is all users and $m$ is the available media. What can be derived from the algorithm is that all the users in the system will be processed, therefore the lower bound of the algorithm can be defined as $\Omega(n)$.

The content-based part is similar but is much more stable in its running time. The user is compared to all media in the system by comparing their vector representations, where the running time is dictated by the size of the media's vector. The algorithm will for each media go through the size of their vector representations, and because of the way the media vectors are constructed, as described in Section \ref{ContentBasedAlg}, the algorithm will run for every media times the longest media vector divided in half.

\[
\frac{mn}{2} = \Theta(mn)
\]

Where $m$ is all media, and $n$ is the size of the longest media vector representation. The running-time is tightly bound and will never deviate from this running time.