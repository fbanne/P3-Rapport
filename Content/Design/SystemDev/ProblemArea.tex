This is an analysis performed in relation to object-oriented analysis and design, and will focus on the problem area of this project. The problem area is the space which a system supervises and represents, in the form of an IT-solution. This analysis is done to describe the reality of the problem area, and to find connections between different entities, and their state.

\textbf{Classes \& Events}

The problem area for this project is  the media interested people, and the media which they wish to consume. It is all about the people involved in consuming different kinds of media, and recommending to other like-minded people, which will be called friends from now on. These people and media is the only which has to be represented in the IT-solution, and is therefore the classes for this problem area.

Events is the instant actions which can be initiated by, or affect, the different classes inside the problem area. The events for this problem area revolve around the consumption of media, the recommendation of media to other people, and people creating and breaking connections with their friends. See Table \ref{EventTable}.

\begin{table}[htb]
\centering
\begin{tabular}{|l|c|c|} \hline
	  & \textbf{Media Interested} & \textbf{Media} \\ \hline
	\textbf{See Media} & X & X \\ \hline
	\textbf{Recommend Media} & X & X \\ \hline
	\textbf{"Get" Friend} & X &  \\ \hline
	\textbf{"Remove" Friend} & X &  \\ \hline
\end{tabular}
\caption{Event Table}
\label{EventTable}
\end{table}

See Figure \ref{ClassDiagram} for a class diagram, which shows connections between different classes, and the cardinality between them.

\begin{figure}[htb]
\centering
\includegraphics[width=0.4\textwidth]{Images/classdiagram.png}
\caption{Classes in the problem area}
\label{ClassDiagram}
\end{figure}

The class diagram shows that the association between media interested people and media, and how multiple people can consume the same piece of media, and how various media can be consumed by a single person. An association from media interested people to itself represents the friends, which these people can make with other like-minded people. The association structure shows that there is no dependencies between these entities, and can exist without each other, which is the correct representation of this problem areas reality.

\textbf{Event Courses}

Next which has to be looked at is the event courses which every object, an instance of class, go through, as it is created. The event course depict the events which the object of a class can be perform or be affected by during its existence, how it changes states inside the problem area, and it may even leave the problem area completely. See figure \ref{Courses}.

\begin{figure}[htb]
\centering
\includegraphics[width=0.8\textwidth]{Images/courses.png}
\caption{Event Courses for the classes}
\label{Courses}
\end{figure}

These event courses shows the independence which these classes has in the problem area, much like the class diagram showed. It shows how all the events are iterative, and be be performed, as long as the object of the class exists. Also shown is various variables which is relevant in the context of a recommendation system. The course table can now be rewritten, to accommodate for the iterative nature of all events in this problem area. See table \ref{UpdatedEventTable}. The multiplication symbol indicates an event can happen multiple times for the instance of a class.

\begin{table}[htb]
\centering
\begin{tabular}{|l|c|c|} \hline
	  & \textbf{Media Interested} & \textbf{Media} \\ \hline
	\textbf{See Media} & * & * \\ \hline
	\textbf{Recommend Media} & * & * \\ \hline
	\textbf{"Get" Friend} & * &  \\ \hline
	\textbf{"Remove" Friend} & * &  \\ \hline
\end{tabular}
\caption{The updated Event Table}
\label{UpdatedEventTable}
\end{table}