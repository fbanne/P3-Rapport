Through the analysis of different approaches to recommendation systems it was recognized that a variation of the hybrid recommendation model was needed. It was decided to approach it very departmentalized. This means the two methods will not directly interfere with each other, which leads to them remaining fairly simple while still getting the benefit of the hybrid model. This however also means the feedback loop will be slightly weaker. With the gained simplicity the slight precision loss from the weaker feedback loop it is, at least for this projects prototype, worth it.

The algorithm ended up being split into three main steps. See Figure \ref{GenRecAlgo}:
\begin{itemize}
	\item Collaborative filtering
	\item Content-based filtering
	\item Merge
\end{itemize}

We will go into detail about the mechanics of the collaborative and content-based filtering algorithms in the following part of this section. The basic idea of this structure is that the two filtering algorithms each pick a list of the best suited media recommendations for the selected user, based on a weight or coefficient. Then in the merge part the two lists are merged into one list, creating a more precise list than the two recommendation forms could do on their own.


\begin{figure}[H]
\centering
\includegraphics[width=1\textwidth]{Images/RecommendationAlgo.png}
\caption{The general structure of the recommendation algorithm}
\label{GenRecAlgo}
\end{figure}


\subsection{Collaborative Design}
\label{CollaborativeDes}
\relinput{CollaborativeDes}

\subsection{Content-Based Design}
\label{ContentBasedDes}
\relinput{ContentDes}