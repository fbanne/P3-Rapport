Entertainment media is a part of almost everybody's lives in some form. It is something we seek out and consume almost every single day, things like books, films, video games or music included. There exist various kinds of ways to spread a piece of media to the public like websites, application, advertising, and social groups. Here can a system which generates recommendations based on various kinds of data like personal information, media data, and social connections can also be included. Either directly or indirectly. This kind of recommendation is tailored towards a certain person, and is there to make them aware that there might be other products which he could be interested in. Like a movie that is similar in genre to what a person previously have watched, or an add-on product to a previous purchase on a retail website.

Recommendation systems have various problems hindering its effectiveness, like data shortage. It is also very crucial that proper weights can be applied for the various kinds of data, so the most important aspect of a piece of media is highlighted for the individual person. An interesting feature could be if it were able to generate recommendations across different kinds of media, e.g. if you liked these books maybe you would like this movie. This would also require a set of suitable connections between them, to create the recommendation. Since these kinds of systems are centred around the user it is also important to include some kind of survey to study people's habits regarding media. These surveys could also provide weights for how important certain data are. These challenges are what creates the base problem for this project.