In order to get the media data which is required for the application to run at an acceptable precision level it was needed to access several open-source media databases. The interface used to communicate with these databases is called an Application programming interface(API). This allows the system to access media data in an easy and controlled way. The term API is used in several different contexts[REF1]. The databases which was  needed to access uses an API integrated with HTTP, so to access different data there had to be sent HTTP-requests. When the database receives these requests and recognises it, the requested data is sent back. Most commonly the data is returned in either XML or JSON (JavaScript Object Notation) format.

We were unable to find any single database that offered all three of the featured media, those being; movies, video games, and books, in this system, so we had to use three different media databases. This was not a big problem since once the code to create and send a HTTP-request was done interpreting the answer was the only additional code needed.  The following three databases were used:

\begin{itemize}
	\item https://www.themoviedb.org/
	\item http://isbndb.com/
	\item http://thegamesdb.net/
\end{itemize}

TheMovieDb and isbndb both required an developer key to access their API. In both instances there was no problem getting the key. You simply had to register a user with the respective services. %--%
Thegamesdb does not require a key at the moment but has announced an update to their API system, which will then require an API key.
The API key is used so the providers can control the users accessing their database, so that  they can stop unintentional or abusive use. Themoviedb and isbndb has set a preemptive limit on the number of requests, albeit for somewhat different reasons. Themoviedb has a limit of 30 requests per 10 seconds[2]. Isbndb has a limit of 500 requests per day. Both of these limits are to control spamming, but the isdbndbs’ limit also has the effect of making them able to make money of, of it. Because you can buy additional requests for a fee.

The limit on isbndb have made it hard to get book data. Since to get books take at least one request per book (it takes an additional request per 20 book when getting books by genre). Themoviedbs’ limit have not cause any trouble.

Both isbndb and thegamesdb returned the media data in XML format, which was easy to handle since a XML-reader is included in the .NET framework. Themoviedb only had the option to return the media data as a JSON object. To read this we found it easiest to use a third-party library[3], this library allowed us to deserialize the json object into a string object. However this still required us to do string manipulation to find the data we needed. The better way to do it is to, in c#, make objects that match the JSON. So that the c# object and JSON have exactly the same fields with the same type and name, that way to can easily parse the JSON object into directly readable data. But because of time restraint and the fact that this codes performance is not detrimental to our program we did not correct this code.



%These media databases are simple used to get mediadata for our system prototype. It could also be a way to continually update our systems collection of media, so it stays up to date, and available to users.