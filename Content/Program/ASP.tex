To make our solution easily available it was decided that the solution needed to be accessible through a browser. Since a specification requirement was that the program had to be written in c\# the obvious choice was building our application on the .ASP framework. Asp is a server-side Web application framework, this means that the majority of the code is executed on the server rather than on the client machine. If, in an asp application you need to execute code on the client side, to for instance check for certain conditions before executing resource heavy code on the server, it can be done through Javascript or JQuery. There are numerous advantages to using the asp framework it takes care of, almost, all of the threading and networking. While we have made our data-structures protected against multi-threading we have not been able to actually test our application with multiple-users. We made our data-structure protected by using a singleton structure which limits an object to never be instantiated more than once, and if done returns the one instance instead of making a new one.

Because all of the code is run server-side and you can manipulate designated html-elements  (designated with the <asp:”...”> tag), which made it simple to make our website dynamic and user friendly. An example of making the website dynamic, in the preference tab when the user changes his preferences the CheckBoxList object has a c\# event handler so that when anything in that object changes the appropriate changes are made on the server.