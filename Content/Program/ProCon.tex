To conclude on the program we looked back upon the requirements described in Section \ref{Requirements}.

The functional requirements have all been followed for the system. The prototype generates recommendations based on a hybrid system, have a GUI which is targeted towards the defined target audience, can easily run on a desktop PC, and on most popular internet browsers. The GUI could have used more user involvement during the design, like showcasing the prototypes to potential users of the system. A rating system was also implemented, which allows to users to rate any media on a scale from 0 to 10.

The solution goals have also mostly been fulfilled during the project. The two forms of recommendations both had certain disadvantages, which made them on their own weak as a recommendation system. Therefore a hybrid of the two were made, to try and counter these weaknesses. Like previously described, it is possible for media across media type to be recommended, especially with the collaborative algorithm. 

One thing which were not fully implemented is to take into account where the recommendation originates from. The idea was that media recommended through friends would receive a boost, and gain an advantage. The structure for this, including a friends list attribute for users, controler, and website tab, have been made, but not implemented. It had a lowered priority, and it was not possible to make time for it in the end. 

Users can alter the recommendation process by how they rate the media. Different ratings will result in different coefficients from the collaborative algorithm, and extraction of recommendations from different users. As long as the users ratings is similar to the ratings of other users, on the same pieces of media, a good coefficient will be generated. The extraction also doesn’t care what kind of media is being recommended, as long as they’ve been rated high. This allows for serendipity. Ratings also have an effect on the vector representation of the user, altering it to their taste, changing the coefficients made with the content-based algorithm. If the user rates a media low, the corresponding elements in the users vector will have their weight subtracted. The same happens if the media item receives a high rating, except the elements in the user vector will have weight added instead.

It can be concluded that, overall, most of the requirements for the program was fulfilled. The program lives up to the expectations which were made in the begining of the project, to a satisfacotyr degree. With the exception of the friendlist implementation in the program.