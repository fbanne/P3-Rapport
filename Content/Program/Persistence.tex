A recommendation system requires a large collection of data of both media and users to generate precise recommendations. In a finished system, some kind of persistence of data would be crucial to the system. Therefore we looked at different ways to persist data, and attempted to implement it into the system.

\subsubsection{NDatabase}

NDatabase is an object database, and is relatively easy to approach. An object database is an object-oriented database, being able to simply store objects, automatically preserving references and does not require creating query strings to store and access the data. Many of the hassles which comes with database is not present in NDatabase and it provides a simpler way to implement a database. \cite{NDatabase}

Unfortunately, NDatabase had some unforeseen problems, which made it impossible for us to effectively use it in the system. NDatabase was very slow, and had problems executing any functions once the data to be stored or retrieved became too high, resulting in Ndatabase running several days with no notable progress.

\subsubsection{Serialization}

To store our data persistently we have used serialization, which is a method of transforming data in objects or data structures to data that can be stored in a file\cite{Serialization}. While it is possible to serialize to many formats we chose binary because it is the simple to deserialize and the only way it differs from other formats is that it is unreadable to humans, which is not that important to us. 

We primarily need to have persistent data for the testing phase of our program. For the final product we would still want to store on disk so that we would be able to dynamically load the data we need into memory. That way we do not have to have all of the data in memory when the program is executed. This would require an actual database and a dynamic loading algorithm to make sure speed is kept up. For the prototype we can decide how much data we want to have by serializing it, and keeping all the data in memory for the lifespan of a test, or trial run of the prototype.