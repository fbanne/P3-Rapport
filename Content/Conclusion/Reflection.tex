This section looks back upon the project and discusses things which could have been done better or differently. It also looks outside the project, at what possibilities are available in society.

We had varying experiences with the API’s. The movie and video game API's had good support albeit the movie API was slow. There was several problems with the book API which prevented us from using it in our system. If our project was to be finished it would be necessary to find alternative sources for the information. IMDB has made contact information available to be used by other commercial interprises if they want access to their huge database. We can only speculate on what would happen if we, as a commercial enterprise, contacted them, but it would probably be possible to lease access to their database. Alternatively it would be necessary to start from scratch and build the required database ourselves, but that would take immense amounts of resources.

When we were trying to build a database locally to hold the information collected through the API’s, we went through several trials and errors. At first we tried to get an early and simple system up and running using NDatabase. It became obvious that NDatabase was slow when it needed to handle anything larger than a private list of phone numbers. We then began researching how to set up a “real” database, but came to the conclusion that it was beyond our capabilities this early in our education, and would take up to much of our available resources. We tried NDatabase one more time, but it became extremely slow and showed signs of stability problems when we started filling it up with movies and users. We ended up using serialization to store the data between test runs.

It felt natural to use the MVC pattern with our ASP.NET application with its three part fragmentation of the system. The MVC has not been fully implemented, as the view sometimes gets information directly from the model, instead of the controller, or through an event. Strictly, this is not allowed with the MVC pattern, but we used a slightly modified version of the MVC pattern for our system.     

When looking at our GUI it is obvious that it would need a heavy reconstruction before publication of the system. Our goal with the GUI was to present it as a prototype. We worked at choosing the colour theme, to avoid distracting the test users unnecessarily. To follow more of the design theories from the Interactive Design course it would be necessary to rework much of the GUI's markup and code behind.

With our main focus on the code and algorithm and only developing the GUI to present the prototype, we missed the chance of using user tests through the different iterations of the program. Especially the collaborative recommendation part is difficult to test without real users, when it works through the non-random choices of other users. If the users in the system are randomly generated, also called dummy-users, the recommendations presented from the collaborative algorithm will also be completely random. In that aspect it was much easier to test the precision of the content-based algorithm. When rating 5 Disney movies at 5 stars we can expect to be recommended another Disney production or similar children/family targeted movies.

When looking at the market for media recommendation solutions we found many other solutions focussing on only one media type, which means that we could fill a niche. One of the few solutions similar to ours was Jybe, which was bought by Yahoo, which shows that Yahoo also believes there is a market for recommending media across different media types. \cite{Jybe}