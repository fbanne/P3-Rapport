This section looks back upon the project, and discusses things which could have been done better, or differently. It also looks outside the project, and what possibilities is available in society.

We had varying experiences with the API’s. The media API, moviedb.com, had good support albeit slow, but we had problems with the other API’s for games and books. If our project was to be finished it would be necessary to find alternative sources for the information. IMDB.com had placed contact information to be used if other commercial interprises wanted access to their huge database. We can only speculate on what would happen if we, as a commercial enterprise, contacted them, but it would probably be possible to lease access to their database. Alternatively it would be necessary to start from scratch and build the required database ourselves. But that would take immense amounts of resources.

Likewise when we were trying to build a database locally to hold the information collected through the API’s, we went through trials and errors. At first we tried to get a beginner friendly system up and running called NDatabase. But it became obvious that NDatabase would be slow when needed to handle anything larger than a private list of phone numbers. We then began researching how to set up a “real” database but came to the conclusion that it was beyond our capabilities this early in our education. We tried NDatabase one more time, but it became extremely slow and showed signs of stability problems when we started filling it up with movies and users. So we ended up using serialization to store the data between test runs.

Design pattern, it felt natural to use MVC with our ASP.NET application with the three parts fragmentation of the program. It makes sense that MVC is often used for programs implementing a GUI.       

When looking at our GUI it is obvious that it would need a heavy reconstruction before publication of the system. Our goal with the GUI was to present it as a prototype. We did a little work with choosing the colour theme and trying to make it as simple as possible, to avoid distracting the test users unnecessary. To follow more of the design theories from the DEB course it would be necessary to rework much of the code. example?  

With our main focus on the code and algorithms and only developing the GUI to present the prototype, we missed the use of using user tests through the different iterations of the program. Especially the collaborative recommendation part is difficult to test without real users, when it works through the non-random choices of other users. If the users in the system are randomly generated dummy-users, then the recommendations presented from the collaborative algorithm will also be completely random. In that aspect it was much easier to test the precision of the content-based algorithm. When rating 5 Disney movies at 5 stars we expect to be recommended another Disney production or similar children/family targeted movies.

When looking at the marked for media recommendation solutions we found many other solutions focussing on only one media type, which means that we could fill a niche. One of the few solutions similar to ours was Jy.be that was bought by Yahoo, which shows that Yahoo thinks theres a market for multi-media recommendations. 