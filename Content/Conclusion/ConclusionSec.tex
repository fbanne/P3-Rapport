This section looks back upon the problem formulation which were made based on the problem analysis, and tries to answer it based on the design, development, and the content which has been processed during the project.

\textit{How can you construct a hybrid recommendation algorithm, based on collaborative filtering and content-based recommendation, to support recommendations across media?}

The hybrid recommendation algorithm have been constructed, using collaborative and content-based filtering, by applying the pearson correlation coefficient and the cosine distance between vectors. Pearson compares users by looking at their shared consumed media and finds users who are alike in their tastes, and recommends other media they have rated high. By utilizing vector representations for all media and users, the cosine distance calculation compares a user with media by their vectors, where each corresponding spot in the vectors is for the same genre, or associated person. The media vector is binary, indicating if the media has that feature. The user can alter their vector by rating media, putting more weight on things they like.

Especially the collaborative part of the algorithm can, theoretically, produce recommendations across different kinds of media, since it only looks at ratings. As long as pearson evaluates them to be similar for any media, the collaborative algorithm can easily generate recommendations across media types. This also means it can have a good serendipity. The content-based algorithm is weaker in this aspect, and is more strong in generating recommendations corresponding to the media type the user has in their medialist. The next step would be to do user tests in the form of a beta test, since the collaborative algorithm cannot be properly tested on randomly generated data.

\textit{How can you present media data to the user, through the graphical user interface?}

Early on it was decided that a website solution were the most logical way to construct this system, since it made sense to have the users in the same environment. Having a central site for it could also help support further work with potential social networking. This can include having the friends list feature implemented, comments, forums, etc. Another reason is that it would have to communicate with a media database most of the time. For this purpose, ASP.NET was chosen to contruct the solution, using both C\#, which was a requirement for the product, and HTML/CSS. 

The GUI was initially designed using a paper and slide presentation prototype, which is the foundation for how the GUI eventually ended up looking. The overall design of the GUI has not changed much since the initial prototype. Some elements have been changed or  removed and theory regarding color has also been applied. The media is presented in similar sized containers with the title and a cover image, if available, for the user to identify the media item. The medialist for each user also shows the rating they have given to the media in the form of stars. The designing process of the GUI could have been much more detailed, involving more theory from the Interactive Design course \cite{DEBBook}, but this was not within the main goals of this project.