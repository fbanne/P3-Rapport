In this closing section we will expand on ideas and possibilities that were not implemented in this project, because of varying constraints. During the research performed for this rapport there were several delimitations we took because of our program being a prototype and time restraints. The ones there will be touch on are:
User testing
Media tags
Implementing other form of media
Algorithm improvements
Friends list
GUI improvements
Smartphone compatibility 

\subsubsection{User testing}
User testing would help us define exactly what parts of our application that are weak. Not only our algorithm but our GUI as well. The testing could be done in multiple stages, because we need different kind of data. First we need to determine how user friendly our GUI is, this could be done with a Usability-test. In such a test you take a group of people and sit down with them while they use your program to perform different pre-defined tasks. Based on how much help they needed and how fast they completed the tasks you can rate your GUI. We would also to test our collaborative recommendation algorithm. This could be done by doing a “closed beta”-test, this is done by allowing a certain number of people onto your services such that you have control over the demographic and amount of people using it. This would help to determine what the amount of users the algorithm actually needs to perform at the desired level, and also if there are specific demographics that it doesn't work particularly well on and adjust it accordingly.

\subsubsection{Media tags}\label{futureWork:Tags}
Media tags is the idea, that like genres you can have a short(mostly one word) that describe the content of the media. This could solve one of the weaknesses of the content-based algorithm, where if the medias you are comparing have a disparity in features the precision of the algorithm lowers. Tags could also help compare media that doesn't naturally have common features, like music or opera. Tags also provides a way for users themselves to define each media item, letting them have more influence, and makes information for each media item more detailed.

\subsubsection{Implementing other form of media}
While games, book and movies provide a good base for prove of concept in a finished product we would like to implement as many media types as possible, among the ones considered are:
Music
Opera
Theater
Cinema-titles
Including a potential social feature that would help you plan cinema goings with your friends and family
Tv and tv-series
Most of these, as mentioned in \ref{futureWork:Tags} have present a problem in the content-based algorithm because they miis common features to compare. But by adding tags and adjusting the algorithm it should be possible to make it work at a desirable precision level. The collaborative algorithm shouldn’t need any tuning specifically to accommodate this problem since it takes no regard to the actual features of the media. Like it was mentioned under the cinema point, social features could be added to the website, taking advantage of social networking. This could be cinema goings which were mentioned, but could also be comments, forums, chats, etc.

\subsubsection{Algorithm improvements}
There are a lot of improvements and tuning that could be made to the recommendation algorithm. The tuning aspect however requires a proper amount of users to see what parts work and which doesn’t. But the one thing that likely will need adjusting are the weights, ie. how much impact preferences have or how much impact having people on your friendslist have. Another thing could be have the vectors in the content-based is accessed, which can be done more efficiently.

Direct improvement, there have been discussions about to major changes that could be made. One is that we could a third form of recommendation. If we add a variation of the ie. demographic recommendation algorithm it would add a lot of precision to the algorithm as a whole. This would however add higher resource requirements to an already fairly heavy algorithm, which potentially could be a problem. Generally adding more kinds of recommendation methods, and different approaches to each method, can help make the recommendation more precise.
Resources and scalability have been discussed and we decided that scalability where not important for a prototype. We did look into a solution on how we could make our algorithm scalable, because right now the collaborative algorithm considers every single user in the database. This, other than slowing down the program also doesn't make sense because not every user are similar and this can be determined by a quicker and more superficial algorithm. Towards this solution we looked into an algorithm called MinHash this is a variation of the LSH(Locality-Sensitive Hashing), which can effectively find "nearest neighbour" for large data sets. Based on that we would be able to group users into smaller groups were all of the users in the group would have some sort of similarity thus not wasting many resources running our collaborative algorithm. This proved to be difficult to implement and therefore were deemed out of scope for this project.


Also, implementing a friends list would give the users another way to affect what recommendations they get. The friends list data structure is ready and it is implemented in the algorithm but it still needs to be integrated with GUI, and the controller for it also needs to be made.

\subsubsection{GUI improvements}
There are several possible improvements that can be done to our GUI, one of the most prominent is our search-function. While it works and looks decent if you keep your search fairly precise. If you get more than eight media returned in your search the navigation becomes quite cumbersome and unintuitive. This need to be improved, since searching and adding media to your user is essential to our system. One way to solve this is to make a proper “page-system” such that you get 20 search results per page, much like most other search functions on the web works.
In this prototype you’re also not able to click on the media in the recommendations tab, while you can click on it in every other tab. This makes for an unintuitive and non-uniform feeling.

In the media list tab you have your current rating of media displayed as stars, it would be better if you could change your rating simply by hovering over the star you want to give it and then clicking. This could be done effectively using the features available in HTML5, which could be applied throughout the interface to make it more efficient. However properly learning HTML5 was outside the scope of this prototype.

\subsubsection{Smartphone compatibility}
In todays world if your service does not have some kind of smartphone capability through an app or having a separate website designed for a smartphone your service is likely not going to perform as well. The current GUI can accommodate a regular browser on a desktop or laptop, which it also would use on smartphones. It is not uncommon for websites to have a different layout depending on the device you access it from. Therefore it is an important part to include in our future work.
