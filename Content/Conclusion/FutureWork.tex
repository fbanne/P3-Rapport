In this closing section we will expand on ideas and possibilities that were not implemented in this project, because of varying constraints. During the research performed for this report there were several delimitations we took because of our program being a prototype. The ones that will be touched on are:
\begin{itemize}
\item User testing
\item Media tags
\item Implementing other form of media
\item Algorithm improvements
\item Friends list
\item GUI improvements
\item Smartphone compatibility 
\end{itemize}

\subsubsection{User testing}
User testing would help us define exactly what parts of our application that are weak. Not only our algorithm but our GUI as well. The testing could be done in multiple stages, because we need different kind of data. First we need to determine how user friendly our GUI is, this could be done with a Usability-test. In such a test you take a group of people and sit down with them while they use your program to perform different pre-defined tasks. Based on how much help they needed and how fast they completed the tasks you can rate your GUI. We would also test our collaborative recommendation algorithm. This could be done by a “closed beta”-test. This is done by allowing a controlled number of people onto your service, such that you have control over the demographic, and the amount of people using it. This would help determining if the algorithm generates recommendations at a desired precision level, and also if there are specific demographics that it does not work particularly well on and adjust it accordingly.

\subsubsection{Media tags}\label{futureWork:Tags}
Media tags is the idea of short words, much like genres, which describes the content of the media item, beyond what the genre is capable of. This could solve one of the weaknesses of the content-based algorithm, where if the media you are comparing have a disparity in features the precision of the algorithm lowers. Tags could also help compare media that does not naturally have common features, like music or opera. Tags also provides a way for users themselves to define each media item, letting them have more influence and makes information for each media item more detailed.

\subsubsection{Implementing other form of media}
While games, books, and movies provide a good base for a prove of concept in a finished product. We would like to implement as many media types as possible, including the following:
\begin{itemize}
\item Music
\item Opera
\item Theater
\item Cinema-titles
\begin{itemize}
\item Including a potential social feature that would help you plan cinema goings with your friends and family
\end{itemize}
\item Tv and tv-series
\end{itemize}
Most of these, as mentioned in \ref{futureWork:Tags} presented a problem in the content-based algorithm because they miss common features to compare. But by adding tags and adjusting the algorithm it should be possible to make it work at a desirable precision level. The collaborative algorithm should not need any tuning specifically to accommodate this problem since it takes no regard to the actual features of the media. Like it was mentioned under the cinema point, social features could be added to the website, taking advantage of social networking. This could be cinema goings which were mentioned, but could also be comments, forums, chats, etc.

\subsubsection{Algorithm improvements}
There are a lot of improvements and tuning that could be made to the recommendation algorithm. The tuning aspect however requires a proper amount of users to see what parts work and which does not. But the one thing that likely will need adjusting are the weights, ie. how much impact preferences have or how much impact having people on your friends list have. Another thing could be how the vectors in the content-based algorithm is accessed, which can be done more efficiently.

There have been discussions about major changes that could be made to improve the algorithm.We could implement a third form of recommendation. If we add a variation of the e.g. demographic recommendation algorithm it would add a lot of precision to the algorithm as a whole. This would however add higher resource requirements to an already fairly heavy algorithm, which potentially could be a problem. Generally adding more kinds of recommendation methods and different approaches to each method, can help make the recommendations more precise.

Resources and scalability have been discussed and we decided that scalability were not important for our prototype. We did look into a solution on how we could make our algorithm scalable. Right now the collaborative algorithm considers every single user in the database. This, other than slowing down the program also does not make sense because not every user are similar and this can be determined by a quicker and more superficial algorithm. Towards this solution we looked into an algorithm called MinHash, a variation of LSH(Locality-Sensitive Hashing), which can effectively find "nearest neighbours" for large data sets. Based on that we would be able to group users into smaller groups where all of the users in the group would have some sort of similarity, thus not wasting many resources running the collaborative algorithm. This proved to be difficult to implement and therefore it was deemed out of scope for this project.

Implementing a friends list would give the users another way to affect what recommendations they receive. The friends list data structure is ready and it is implemented in the algorithm, but it still needs to be integrated with the GUI and the controller for the friends list to be fully implemented.

\subsubsection{GUI improvements}
There are several possible improvements that can be done to the GUI, where one of the most prominent is the search-function. It works and looks decent if you keep your search fairly precise, but if you get more than eight media returned as result the navigation becomes quite cumbersome and unintuitive. This needs to be improved, since searching and adding media to the medialist is essential to our system. One way to solve this is to make a proper “page-system” such that you get 20 search results per page, much like most other search functions on the web works.
In this prototype you are also not able to click on the media in the recommendations tab, while you can click on it in every other tab. This makes for an unintuitive and non-uniform feeling.

In the medialist tab you can see the rating of each media in the medialist in the form of stars. One improvement could be being able to change the ratings simply by howering over the stars. This could be done effectively using the features available in HTML5, which could be applied throughout the interface to make it more efficient.

\subsubsection{Smartphone compatibility}
In todays world, if your service does not have some kind of smartphone capabilities through an app or having a separate website designed for smartphones, your service is likely not going to be received as well. The current GUI can accommodate a regular browser on a desktop or laptop, but it cannot support the difference in screen size or added interactive features on a smartphone. It is not uncommon for websites to have a different layout depending on the device you access it from. Therefore it is important to include smartphone capabilities in our future work.
