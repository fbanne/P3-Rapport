\section{Future Work}
\label{conclusion:FutureWork}

In this closing section we will expand on ideas and possibilities that were not implemented in this project, because of varying constraints. During the research performed for this rapport there were several delimitations we took because of our program being a prototype and time restraints. The ones there will be touch on are:
\begin{itemize}
	\item User testing
	\item Media tags
	\item Implementing other form of media
	\item Algorithm improvements
	\item Friends list
	\item GUI improvements
	\item Smart phone compatibility 
\end{itemize}

\subsubsection{User Testing}
User testing would help us define exactly what parts of our application that are weak. Not only our algorithm but our GUI as well. The testing could be done in multiple stages, because we need different kind of data. First we need to determine how user friendly our GUI is, this could be done with a Usability-test. In such a test you take a group of people and sit down with them while they use your program to perform different pre-defined tasks. Based on how much help they needed and how fast they completed the tasks you can rate your GUI. We would also to test our collaborative recommendation algorithm. This could be done by doing a “closed beta”-test, this is done by allowing a certain number of people onto your services such that you have control over the demographic and amount of people using it. This would help to determine what the amount of users the algorithm actually needs to perform at the desired level, and also if there are specific demographics that it doesn't work particularly well on and adjust it accordingly.

\subsubsection{Media Tags}\label{futureWork:Tags}
Media tags is the idea, that like genres you can have a short(mostly one word) that describe the content of the media. This could solve one of the weaknesses of the content-based algorithm, where if the medias you are comparing have a disparity in features the precision of the algorithm lowers. Tags also help compare media that doesn't naturally have common features, like music or opera.

\subsubsection{Implementing other form of Media}
While games, book and movies provide a good base for prove of concept in a finished product we would like to implement as many media types as possible, among the ones considered are:
\begin{itemize}
\item Music
\item Opera
\item Theatre
\item Cinema-titles
\end{itemize}

Including a potential social feature that would help you plan cinema goings with your friends and family
Tv and tv-series
<<<<<<< HEAD
Most of these, as mentioned in \ref{futureWork:Tags}, on page \pageref{futureWork:Tags}, have present a problem in the content-based algorithm because the miis common features to compare. But by adding tags and adjusting the algorithm it should be possible to make it work at a desirable precision level. The collaborative algorithm shouldn’t need any tuning specifically to accommodate this problem since it takes no regard to the actual features of the media.
=======
Most of these, as mentioned in \ref{futureWork:Tags} have present a problem in the content-based algorithm because they miss common features to compare. But by adding tags and adjusting the algorithm it should be possible to make it work at a desirable precision level. The collaborative algorithm shouldn't need any tuning specifically to accommodate this problem since it takes no regard to the actual features of the media.
>>>>>>> 18dd4fbda198405c227447004eb2ce3891c14d03

\subsubsection{Algorithm Improvements}
There are a lot of improvements and tuning that could be made to the recommendation algorithm. The tuning aspect however requires a proper amount of users to see what parts work and which doesn't. But the one thing that likely will need adjusting are the weights, i.e.. how much impact preferences have or how much impact having people on your friend list have.

Direct improvement, there have been discussions about to major changes that could be made. One is that we could a third form of recommendation. If we add a variation of the ie. demographic recommendation algorithm it would add a lot of precision to the algorithm as a whole. This would however add higher resource requirements to an already fairly heavy algorithm, which potentially could be a problem.

Also, implementing a friends list would give the users another way to affect what recommendations they get. The friends list data structure is ready and it is implemented in the algorithm but it still needs to be integrated with GUI.

\subsubsection{GUI Improvements}
There are several possible improvements that can be done to our GUI, one of the most prominent is our search-function. While it works and looks decent if you keep your search fairly precise. If you get more than eight media returned in your search the navigation becomes quite cumbersome and unintuitive. This need to be improved, since searching and adding media to your user is essential to our system. One way to solve this is to make a proper “page-system” such that you get 20 search results per page, much like most other search functions on the web works.
In this prototype you're also not able to click on the media in the recommendations tab, while you can click on it in every other tab. This makes for an unintuitive and non-uniform feeling.

In the media list tab you have your current rating of media displayed as stars, it would be better if you could change your rating simply by hovering over the star you want to give it and then clicking. This could be done effectively in HTML5, however learning HTML5 was outside the scope of this prototype.

\subsubsection{Smart Phone Compatibility}
In today's world if your service does not have some kind of smart phone capability through an mobile-app or having a separate website designed for a smart phone your service is likely not going to perform as well. Therefore it is an important part to include in our future work.

