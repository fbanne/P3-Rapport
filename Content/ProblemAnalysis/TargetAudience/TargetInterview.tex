We can see from the interviews that younger people seem to have more interest in our proposed project. One reason this might be the case is because younger people already use the internet to find new media to consume, and most also use other recommendation systems. While our sample size isn’t big enough to make a definite conclusion it helped us define questions for our questionnaire, such at the what age groups we should split people into. 

The interviews also hint that peoples’ current employment status doesn’t have a significant effect on their interest in our project. There was one person that outright said she had no interest in our project, and had only one relevant difference from the other people we interviewed which was her age. So that doesn’t signify a trend based on employment. And the other 10 interviewees were a mix of unemployed, employed and had various educational levels.

To note about privacy which was a minor focus of the interviews we see that age doesn’t really impact concerns on privacy. While younger people were a bit more specific on the technicals and on the different forms of data, they all agreed that they wanted to choose what data the company keeps on them. They also agreed that the data, which the company does not need to run their service, should never be stored.

The interviews not only helped us confirm that there is an interest in the project but also to specify our target audience. We now know that our focus should be towards the age group 18-50, based on the responses we got.