For this project there has been prepared several requirements for the product. The product would have to be in the form of a program which allows a person to list media which he or she has consumed and then receive recommendations for new pieces of media, based on previous behavior. The system should be approachable but will have a focused target group from the late teen years into the thirties. The system should function as a way to manage and keep track of previously consumed media but also as a way for the person to discover new media, based on the recommendations the system generates. Besides the collaborative recommender we decided to develop a hybrid with a content-based recommender. The product is also going to prepare a fitting graphical user interface, in the form of a web page.

\begin{itemize}
	\item Functional Requirements
	\begin{itemize}
		\item It has to be possible for the prototype to generate recommendations based on a hybrid recommender system.
		\item The prototype should have a fitting graphical user interface for the chosen target audience.
		\item The prototype has to be usable and as a minimum be able to run locally on a regular desktop PC with an internet browser, e. g. Google Chrome.
		\item The prototype should implement a rating system, where users indicate their satisfaction with a piece of media based on a scale from e. g. 0 to 10.
	\end{itemize}
	\item Non-Functional Requirements
	\begin{itemize}
		\item The prototype, together with the associated project report, have a shared deadline the 20th of December 2013.
		\item The prototype has to be written in the C\# programming language, following object oriented programming.
	\end{itemize}
	\item Solution Goals
	\begin{itemize}
		\item The hybrid recommendation algorithm should help the system create more precise recommendations.
		\item It should be possible for recommendations to be given across different media, e.g. from data regarding movies into a video game recommendation.
		\item Recommendations should have additional weights added, to favor certain aspects of a media like which person it originated from or preferences, based on survey results.
		\item It should be possible for the users themselves to alter the recommendation process through preferences and indicating unwanted factors in a piece of media.
	\end{itemize}
\end{itemize}