In the content-based system the data used to generate the recommendations comes from item descriptions, tags or significant words in a text. The system then builds a user profile of user interests based on that data. Through matching the user profile against other items the system can then make a relevance judgment that represents the users level of interest in the items. (Recommender Systems Handbook)


\subsubsection{The Demographic System}

In the demographic recommendation system the data used to generate the recommendations comes from demographic data such as age, gender and occupation. The demographic data is used to generate the user profile and it is then possible to recommend items to different demographic niches based on the combined user ratings in those niches. (wiki recommender)


\subsubsection{The Knowledge-based System}

The knowledge-based system is used to solve complex problems and was first developed by artificial intelligence researchers. The system is most commonly constructed with one or both a knowledge base  and an inference engine. The inference engine represents logical assertions and conditions about the world usually as IF-THEN rules.

Early knowledge based systems was used as expert systems. An example was Mycin that was used in medical diagnosis. Later they have been adapted to the internet and is often used to classify objects in complex and unstructured data.

But as the knowledge based system works from assertions and conditions that all have to be programmed the system hits a knowledge engineering bottleneck as the amount of data it has to handle grows. (wiki knowledge-based systems)

