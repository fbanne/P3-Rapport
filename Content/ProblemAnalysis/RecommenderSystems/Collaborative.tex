In the collaborative system the data used to generate the recommendations takes into account models made on other users. The theory behind the collaborative systems is that "users with similar interests are likely to find the same resources interesting for similar information needs." [REF]

If you take a small group of people interested in movies who meet each other regularly, eg. at work or similar. They will talk about what new movies they have seen and recommend the ones they liked to the others in the group. After watching some of the different recommended movies each of members of the group will get a sense of who in the group recommend movies they like, who recommend movies they don't like and who recommend a mix.

In short they have identified what members of the group have similar interests so they can get recommendations from them.

What the collaborative system does is take that idea and apply it to a computer system with thousands of people. And with the speed of computers it can be made in real time and with a much higher precision than with only the small group of coworkers.