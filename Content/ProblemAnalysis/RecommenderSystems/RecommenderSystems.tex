%Recommender Systems Section
Through the surveys we found that there is an interest in getting recommendations from different sources. Especially from friends and family and also through similar items like movies with some of the same features such as an actor or for a game a specific developer who previously have made highly praised games. But what is recommendations exactly and how do they work.

According to \cite{TheAdaptiveWeb} there are 3 ways users retrieve desired information on the internet; searching by browsing, searching by query, and through recommendations.

When browsing, users go through webpage's sequentially following hyperlinks. It can be useful if you have already found the webpage containing the data you were looking for or the page that has the required hyperlinks. An example is Wikipedia, where a lot of the expressions on a page are also hyperlinks to another page further explaining the expression, for anything from the name of a philosopher to a mathematical equation. But the approach is not useful if the page on which the information is located has not yet been found.


A way to locate the correct page on the web is through querying a search engine. But according to \cite{TheAdaptiveWeb} queries through search engines work best when you know the exact and right wording for the item or webpage needed. Too specific problems are polysemy; the existence of multiple meanings for a single word and synonymy; the existence of multiple words with the same meaning. This becomes especially apparent when you get stonewalled in a querying system, which means no results could be returned based on the criteria.


The last option for retrieving the desired information is through recommendations. Recommendation-based systems can suggest items, like products from a catalogue, movies or news articles for the user.


\subsection{Background}
\label{Background}
\relinput{Background}

\subsection{Business Perspective}
\label{BusiPers}
\relinput{BusinessPers}

\subsection{Recommendation Systems Overview}
\label{Overview}
\relinput{Overview}