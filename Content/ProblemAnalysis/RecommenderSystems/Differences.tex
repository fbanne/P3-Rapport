All the recommendation systems have different weaknesses and strengths. \cite{TheAdaptiveWeb}.

If we look at the learning based systems they suffer from cold start problems, which means they have difficulty handling new users and items. They need information about the user to work properly. 

Another problem is stability vs. plasticity, once the system has learned a good deal about a user, it has difficulty changing its results. If a user decides to become a vegetarian the food shopping site he uses will keep giving him meat recommendations for a long time before reflecting his new eating habits. To try and contain that problem some developers include temporal discount in their systems, to cause older ratings to have less influence, but by doing that they risk losing information about sporadically exercised behavior. The knowledge based systems isn't learning based and is therefore the only one of the four common systems that doesn't suffer from that problem.

A huge strength for the collaborative and demographic systems is their unique capacity to identify cross-genre niches, they can sometimes recommend an item of interest to the user which is outside the users normal areas of interest. The knowledge-based system can also do this to some degree but only if the associations have been identified ahead of time and programmed into the system.

The hybrid system tries to meet some of those shortcomings by combining two or more of the systems. 
