This section describes what possible similarities in attributes there can be between different kinds of entertainment media, as a way to recommend content across them. This is going to look at the content-based recommendation possibilities, which is purely based on data and attributes about the content, like genre and involved people, rather than user and friend ratings.

First we have books, movies, and videogames. These three have a clear connection as they can, and often is, adapted into each other. There exist many movies which is based on a book, like the ‘Lord of the Rings’ trilogy, and various videogames based on books, like ‘The Witcher’ and ‘Metro 2033’ series. For videogames, the other way around is usually a result of rising popularity, and uses novels and books for world-building and secondary plot-lines. Genre is also a clear connection, as all three has the same genres, like crime and science fiction, prevalent through them. Associated people can also provide another connection, like a script writer or director, who previously have worked on both movies and videogames. Television shows can also be fitted into these three, especially movies, who shares many of the same similarities.

For something like music, there isn’t any clear connections to any of the previously mentioned entertainment media. Music does appear in the visual entertainment media, like movies and videogames, but mainstream music usually only appear in some movies, and more often than not, movies and videogames has their own scores. Genre connections doesn’t apply either, as music has its own set of genres, like rock and pop. For associated people there can be some connections, like if a piece of music shares a composer with a score from a movie. When it comes to books, which doesn’t have any music attached to its form of entertainment media, there is even less suitable connections to be made. It could be argued that there does exist connections between music and books, because for every topic in existence, there will be books about it. Music could appear in books as a story element, or in educational music books. There could also exist connections with biographies depicting the lives of musicians. This is still problematic as these connections are quite niche. This project’s topic is also about entertainment media, so something like educational media does not apply here. 

A majority of people do listen to some kind of music though, so it could possible, or more suitable, to link users with similar taste in music. Through that music can be incorporated, together with the connections that do exists. This also applies to any other which has already been mentioned, and so, their recommendations can further be reinforced and improved together with their own connections.

Another thing that could provide usable connections can be indirect connections. For example, if you have a certain book, and it has a movie adaption. This movie uses a certain piece of music, which turns out to have variable that match with other media items which the same user likes. This can provide a whole new dimension of connections and recommendations, but also make the recommendation process much more complex.

There exist numerous ways to make connections between the books, movies, and video games, which can be used as parameters for generating content-based recommendations. For music though, there is a clear lack of connections suitable for generating proper recommendations, as it shares minor similarities with movies, videogames, and books. The connections that does exist are quite niche, and may serve more of an educational purpose, than entertainment. It is quite clear though, that music is something a majority of people consumes. So if it can’t get as many connections to other kinds of media, it could work in a more social kind of way, where it takes personal information into a higher consideration, and links people who have a similar taste in music, and though that might even recommend a movie or a book.