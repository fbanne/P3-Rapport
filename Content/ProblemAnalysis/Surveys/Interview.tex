Before we decided how to create a media recommendation product, it would be a good idea to figure out if there really are a need for such a system. Because of that it was decided that there had to be constructed an interview and a questionnaire, which could tell us if there are any people who would be interested in using a media recommendation website. In addition to the questions regarding media recommendations, there was also some questions regarding user privacy rights.

The project group were split into teams. The first team went to GameStop and Fona, the second team went to the library, and the third team went to the cinema. The only requirement there was, was that children were not going to be interviewed. The interviewing was conducted by approaching people in the target area and interviewing them about their media habits. The people who were interviewed did not know the purpose of the survey, until the last question was asked, so that what the project was about did not affect their answers. 

15 people were interviewed all with different ages and jobs/educations. The goal was to find similarities that the interviewed agreed on regarding media recommendation, and possibly other information, such as ideas for a possible product. It should also clarify whether or not a questionnaire were required, and if it were, help formalize what it had to achieve.

There were a lot of similarities between the different people who were interviewed. One thing that people agreed upon was that they already use similar webpages to what this project had in mind. Another thing that was common was that the majority of the people got recommendations from their friends and weighted them higher than critics and other people. However, one surprising thing was that they also weighed their own opinion higher than their friends, even though they have not seen that movie or book before. 

The oldest person asked was a 52 year old woman, who admitted she most likely would not use our proposed solution. Besides her there was an agreement among people that such a solution would be a good idea and that they would most likely use it. This might suggest that a possible solution would be more suitable targeted at a lower age group. As expected, regarding the questions on user privacy and rights, most people were less comfortable with their personal information being available to other people, as the amount of information became more personal. From the interview responses it could be seen that most people tend to first take their own oppinion into account and then those of their friends when picking new media which translates to the personalized and social approach to recommendations, rather than through an item recommendation. 


The following is a list of other tendencies found when analysing the interview answers:

\begin{itemize}
	\item The majority of people were mostly interested in movies and music as their entertainment media.
	\item You should be able to see how friends have rated different media.
	\item A ‘Mood System’ were suggested as a recommendation feature based on your mood.
	\item Recommendations from advertisements or newspapers is used, but is less convenient because it is usually not available on the fly.

\end{itemize}
