Before we decide how to create a media recommendation product, it would be a good idea to figure out, if there really is a need for such a system. Because of that it was decided that there had to be constructed a interview-questionnaire, which could give an idea on if there are any people who would be interested in using a media recommendation website. Besides questions regarding media recommendations, there was also some questions regarding user privacy rights

The project group were splitted into into teams and then discussed where it would be suitable to go, with the area our project is about in mind. The first team went to GameStop and Fona, the second team went to the library, and the third team went to the cinema. The only requirement there was, was that children were not going to be interviewed. The interviewing was conducted with a member of the group approaching a person and ask if they were interested in giving their opinion about different kinds of media. The people who were interviewed did not know the purpose of the questions, until the last question was asked, to avoid people making their opinions out from the idea of a media recommendation system. 

Altogether, 15 people were interviewed, all with different ages and jobs/educations. After interviewing, the interview groups met up, and began analyzing all the answers that they had collected from the interviewees. The goal was to find things that the interviewed people agreed on regarding media recommendation, and possibly other information, such as ideas for a possible product. It should also help make clear whether or not a questionnaire survey were required, and if it were, help formalize what it had to achieve.

There was a lot of similarities between the different people who were interviewed. One thing that people especially agreed on, was that they already use similar webpages to what this project had in mind. Another thing that was common about the answers, was that the majority of the people got recommendations from their friends and weighted them higher than critics and other people. However, one surprising thing was that they also weighed their own opinion higher than their friends, even though they haven’t seen that movie or book before. 

The oldest person who was asked during these interviews was a 52 year old woman, who admitted she most likely would not use such a solution. Besides her there was a pretty large agreement by people, that such a solution would be a good idea in some way and they would most likely use it. This might suggest that a possible solution would be more suitable targeted at a lower age group. As expected, regarding the questions about user privacy and rights, most people were less secure with their personal information being available for other people, as the amount of the information became more severe. From the interview responses, it could be seen that most people tend to pick up recommendations based on a personalized and social way, rather than through an item recommendation. 

The following is a list of other aspects which were picked up during the interviews:

\begin{itemize}
	\item The majority of people were mostly interested in movies and music as their entertainment media.
	\item It was desired that it you should be able to see how friends have rated different kinds of media.
	\item A ‘Mood System’ were suggested, a recommendation feature based on your mood.
	\item Recommendations from advertisements or newspapers is used, but is less convenient because it is usually not available on the fly.

\end{itemize}
