For a recommendation system it is required to collect data about the intended recipients of the recommendations. The recommendations become better and more precise as the amount of data grows, so a recommendation system will usually try to gather as much data as it possibly can. This is also called data profiling, where the system categorizes people depending on their taste in media, personal information like age and sex, and their social connections. In other instances it also includes variables like previous purchases, view history, tags, and keywords extracted through text analysis. All of these might be more revealing, and raises privacy concerns.\cite{UserRights2}

With this amount of data available to a project like this, there will be privacy concerns. Users will most likely be required to register a profile or account to utilize the system, and will have to hand over personal information for verification and recommendation purposes. This is of course not any different from many other systems that does similar things. In this case we are talking about a recommendation system, which poses new challenges in this aspect. Because recommendations can be based on the data of other people, it would make it possible to deduce connections to those said people. Especially if it is through some uncommon element, like an obscure cambodian film, where fewer people are connected. This can ultimately lead to personal information being exposed. This risk is increased if the person also has access to the database and can place queries.\cite{UserRights1}

This is a challenge for any recommendation system, but its also a complex and difficult problem to solve. It depends on exactly how this projects product is going to be constructed, which is not yet formalized, to properly answer these questions. It is therefore out of this projects scope.

We conducted some interviews where we asked potential users several questions regarding the project, including questions regarding how much information they would be okay with being available either to a company or the public. See Section \ref{Interview}. The questions was asked such that it was apparent exactly where people’s threshold would be. Most people were okay with their contact information being available, as it was most likely already available in some way or form. When it came to their more personal information, like interests and age, they were uncertain. Almost no one was okay with having their very personal information, e.g. emails, chats, and pictures available. An interesting feature a lot of them mentioned were an option to choose whether this information is shared. This could be a more user-engaging way to collect information for the recommendation system, and solves most problems regarding user rights, as the user decides for themselves what they want to share.
