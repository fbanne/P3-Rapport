In the problem analysis, different media recommendation systems was discussed. It was argued that there did not exist many media recommendation systems that could recommend across different media and so this became the main goal of this project. Through statistics and by conducting surveys, it was determined that the majority of people use more than one kind of media regularly and that there were interest in our proposed solution. This also produced crucial data, which can be used to weigh the importance of various attributes in media and determine which other users in the system should have more influence on the recommendations. Various media connections were also explored, to try and define connections between different media, which can be exploited in the recommendations process. Based on this the following problem formulation was made:




\textit{How can you construct a hybrid recommendation algorithm, based on collaborative filtering and content-based recommendation, to support recommendations across media? How can you present media data to the user through the graphical user interface?}

\subsection{System Definition}
\label{SysDefinition}
\relinput{Systemdefinition}
