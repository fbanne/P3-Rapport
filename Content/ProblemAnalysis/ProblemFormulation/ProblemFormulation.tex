In the previous problem analysis, there has been argued for various kinds media recommendation systems. It was argued that there didn’t exist much when it came to media recommendations who could recommend different kinds of medias, and so this became the main goal of this project. Through statistics, and later by conducting surveys, it was determined that most asked people used more than one kind of media regularly, and saw viability in the proposed system. This also produced crucial data, which can be used to weight the importance of various attributes in media, and determine which other users in the system should have more influence on the recommendations. Various media connections was also explored, to try and build bridges between different media, which can be exploited in the recommendations process. Based on this, and the analysis, the following problem formulation has been made:

\begin{itemize}
	\item How can you recommend pieces of media across different kinds of media, based on similarities in general item information, and similarities with other people? 
	\item How can you construct a hybrid recommendation algorithm, based on collaborative filtering and content-based recommendation, to support recommendations across media? 
\end{itemize}

\subsection{System Definition}
\label{SysDefinition}
\relinput{Systemdefinition}
