Amazon is the worlds largest online retailer. It began as an online bookstore but then it soon began selling several other things like movies, electronics, and clothes.

Amazon's system uses the combination of the three approaches. Which means the system is based on the individual behaviour and either the item itself or behaviour of similar users. Amazon have a “Customers Who Bought This Item Also Bought” beneath the item you are viewing, which is a social and personalized recommendation. Customer reviews are also available to be viewed for the item, where the customer can rate and comment on the item. And based on the reviews it shows the top 3 most similar statements. Amazon also gives you other recommendations based on what your history has been, which is personalized recommendation.

\subsubsection{YouTube}

YouTube is a popular video sharing website which is owned by Google. YouTube's system uses the personalized recommendation. YouTube takes your history, and other activities on the site, to give you recommendations for both channels and videos. So basically whatever video you click on will be placed in your personal history list and based on that it will give you recommendations.

The problem with YouTube is that even if you have only watched one second of a certain video, it will be included in the recommendations to come. Unwanted videos can easily be recommended because of this. Channels also gets recommended even after only watching a single video from the channel.

\subsubsection{IMDB}

Internet Movie Database(IMDB) is a database with movies. It shows information about new movie releases and you can watch trailers from movies before you decide to watch them. IMDB is a kind of  personalized recommendation and social recommendation system. 

It also has recommendations based on what you have previously watched and rated. It is a social recommendation system, because it is based on what other people have previously watched. There is no friends list so there is no control regarding what people these social recommendations are based on. The personalized recommendation system also has some flaws. As the project group members tried testing it as a member and not a member of the site, it gave the same recommendations and showed movies that had already been watched and rated by the user.

\subsubsection{LinkedIn}

LinkedIn is a social networking website for professional occupations. It gives people the possibility to find, and be found, for projects or work based on ones skills, previous experience, and descriptions from other users. It uses text analysis to find certain keywords like “trustworthy” or “dedicated” to highlight a person's abilities. LinkedIn also has a “apply from LinkedIn” button where users can apply for a position in a company through their LinkedIn profile. 

The system is based on item recommendation as it is the descriptions, skills, and keywords that is used for the recommendation. When a user searches for a person or a company it is a personalized recommendation. The system could also be a combination recommendation system. There can occur problems, if someone writes wrong descriptions about other users, both if it is intentionally better or worse.