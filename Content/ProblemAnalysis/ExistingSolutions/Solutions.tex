Amazon is the worlds largest online retailer. It began as an online bookstore but then it soon got several other things, like movies, electronics and clothes etc..

Amazon's system uses the combination of the three approaches. Which means the system is based on the individual behaviour, and either the item itself or behaviour of similar users. Amazon have a “what other customer also bought” under the item you are viewing, which is a social and personalized recommendation. A customer review is also possible to be viewed for the item, where the specific customer is available to rate and comment on the item. And based on the reviews it shows the top 3 most similar statements. Amazon also gives you other recommendation based on what your history has been, which is personalized recommendation

\subsubsection{YouTube}

YouTube is a popular video sharing website which is controlled by Google. YouTube's system uses the personalized recommendation. YouTube takes your history, and other activities on the site, to give you recommendations, on which it gives recommendations for channels and videos. So basically whatever video you click on, it will be placed in your personal history list and based on that it will give you some recommendation.

The problem with YouTube is that even if you only have watched one second of a certain video, it will be included in the recommendations to come. Unwanted videos can easily be recommended because of this. Channels also gets recommended even after only watching a single video from the channel.

\subsubsection{IMDB}

Internet Movie Database(IMDB) is a database with old and new movies. It gives information about new movie releases and you can watch trailers from the movies before you decide to watch it. IMDB is a kind of  personalized recommendation and social recommendation system. 

It also has recommendations based on what you’ve previously seen and rated. The system is to a certain degree, flawed. Despite being sort of a social recommendation, because it is based on what other people have previously seen, there is no friend list or similar, so there is no control regarding what people these social recommendations is based on. The personalized recommendations also has some flaws. As the project groups members tried seeing what it would generate, as a member and not a member of the site, it gave the same recommendations, and showed movies that have already been seen and rated.

\subsubsection{LinkedIn}

LinkedIn is a social networking website for professional occupations. It gives people the possibility to find, and be found, for projects and/or work based on ones skills, previous experience, and descriptions from other users. It uses text analysis to find certain keywords, like “trustworthy” or “dedicated”, to highlight a person's abilities. LinkedIn also has a “apply from LinkedIn” button, where users can apply for a position in a company, through their LinkedIn profile. 

The system is based on item recommendation, as it is the descriptions, skills, and keywords are extracted about the user, that is used for the recommendation. When the user itself search for a person or a company that will be a personalized recommendation. The system could also be some kind of hybrid recommendation system. There can be some problems, if someone begins to write wrong descriptions about another user, both if it is intentionally better or worse than it really is.