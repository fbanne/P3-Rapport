This section is going to make a boundary on the width of this project’s problem area, based on the previous problem analysis. This is done to limit the project to what is the most important, which is based on survey results, and other information collected or augmented throughout the problem analysis.

Through the surveys that was done, both the interview and the questionnaire, one of the goals was to find out what kind of recommendation has the highest appeal for most potential users. There was three kinds of recommendations which was evaluated; Personalized, social, and item. Looking at responses generated with the surveys, it became apparent that social was the dominant kind. Followed by personalized, realized with recommendation sites like IMDB, and lastly item recommendation. Recommendations based around the person itself, and its nearest social bonds, seems to be the most common kind of recommendation. Based on this, the project will be limited to focus on social and personalized recommendations. For a completely finished recommendation system, a hybrid of all three should be considered, including item recommendation, as it has its own useful features.

We then looked at how the recommendation systems work in theory and to be able to provide the desired social and personalized recommendations a collaborative recommender was chosen. The collaborative system is likewise very good at recommending different types of media because it only look at users and their ratings, meaning it doesn't differentiate between a movie and a piece of musik.  But the collaborative system alone suffers from issues when there's a large database of media but scarce ratings, the cold start issue. A very common hybrid of recommendation systems today is a mix between a collaborative and a content-based system and that would minimize the cold start issue making it possible for the system to recommend items that hasn't yet been rated or with only a few ratings. Other great properties of a collaborative, content-based hybrid is that they have overlapping sources as both use user ratings and history. Therefore a content-based recommender was also chosen. 

Through a form of weighted hybrid system the user profile customization can be expanded to include preferable genres and other features giving the user a little more power over what recommendations will be given. 

Although the model-based system sounds better with more scalability we will make a memory based system to stay within the scope of our project, meaning that most of the calculations will happen when a recommendation is requested.

Regarding what kinds of media should be included in the project, the answer is not quite as clear. It was augmented for in section [REF] that there exist clear connections between books, movies, and video games, while music was the odd one out between the four. But, since it has already been made clear that the project will be based on social and personalized recommendation, this becomes less of an issue. With this the recommendations will not be generated purely on attribute connections between media, but by data collected from users and their connections to other users. Survey results from the questionnaire also back up this stance. Despite this, it is necessary to limit the project to gain more simplicity, as recommender systems is already quite complex. In a finished product it is expected that all entertainment media forms is included, but for this project, music is going to be excluded for initially reducing the workload.

Early in the project it was considered that issues like user privacy and rights was to be tackled in this project, since a recommendation system uses personal information regarding its users to generate recommendations. As expected, the interview survey further backed up this stance, as most people who were asked were less secure with their personal information being used without their consent. A possible method was suggested in section [REF], like giving the user themselves the possibility to choose whether or not they give consent to using their personal information, with less accurate recommendations as a consequence if they should decline. Despite this, the problem with user rights is difficult to answer, as it is quite an ethical problem, and therefore out of this project’s scope. For this reason, user privacy and rights will not be looked at further in this project.

The precise target audience for this project is hard to set, as entertainment media is something almost any age group consumes to some degree. What has been determined is that the project will not cater to children, and therefore no entertainment media specifically targeted towards children. This is done to prevent making the designing process more difficult, as the product would have to cater to children otherwise. You could imagine that the site could be used by parents to find children entertainment media to enjoy with their children, but for now, the project will be limited away from this kind of feature. People with physical disabilities, or other disadvantageous which prevents normal use of personal computers, is also not included in the target audience. This is done for the same reason as children.

A more precise age group for the target audience can be specified by looking through survey results. In the interview survey, a 52-year old interviewee showed disinterest in the overall concept of this project, and was less aware of recommendation possibilities that can be generated through websites like IMDB. This trend became more apparent depending on how old the interviewee was. The questionnaire survey showed, as expected, that all age groups has some form of interest in entertainment media. Still, the exact age group is hard to set, but considering what has already been stated regarding children, the lower bounds of the target age group can be set to around 17 years. The upper bounds is less clear, and can be set from anywhere between the 30’ies to the 40’ies. These bounds is put in place to figure out who the possible solution should be designed for, rather than what entertainment media can be excluded, as even a fifty year old movie can still be popular today.