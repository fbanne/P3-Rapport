The boundary of this projects problem area, based on the problem analysis, was made to focus the project towards certain areas. This is done to limit the project to what is the most important, which is based on survey results and other information collected or augmented throughout the problem analysis.

A goal of our surveys was to find out what kind of recommendation has the highest appeal for most potential users. There was three kinds of recommendations which was evaluated; Personalized, social, and item. Looking at responses it became apparent that social was the dominant kind. Followed by personalized, realized with recommendation sites like IMDB, and lastly item recommendation. Recommendations based around the person himself and his nearest social bonds seems to be the most common kind of recommendation. Based on this, the project will be limited to focus on social and personalized recommendations. For a completely finished recommendation system a hybrid of all three should be considered, including item recommendation as it has its own useful features.

We then looked at how the recommendation systems work in theory to be able to provide the desired social and personalized recommendations a collaborative recommender was chosen. The collaborative system is good at recommending different types of media because it only look at users and their ratings, meaning it doesn't differentiate between a movie and a piece of musik.  But the collaborative system alone suffers from issues when there's a large database of media but scarce ratings, the cold start issue. A common hybrid of recommendation systems today is a mix between a collaborative and a content-based system that would minimize the cold start issue. And make it possible for the system to recommend items that has not yet been rated or have only a few ratings. Other desirable properties of a collaborative, content-based hybrid is that they have overlapping sources as both use user ratings and history. Therefore a content-based recommender was also chosen. 

Through a form of weighted hybrid system the user profile customization can be expanded to include preferable genres and other features giving the user more power over what recommendations they recieve. 

Although the model-based system sounds better with more scalability the memory based system was chosen to stay within the scope of our project, meaning that most of the calculations will happen when a recommendation is requested.

Regarding what media should be included in the project, the answer is not clear. It was argued in section \ref{Connections} on page \pageref{Connections} that there exists connections between books, movies, and video games while music was the odd one out between the four. Since it has already been established that the project will be based on social and personalized recommendations this becomes less of an issue. With this the recommendations will not be generated purely on attribute connections between media, but by data collected from users and their connections to other users. Survey results from the questionnaire also back this up. Despite this it is necessary to limit the project to keep it simpler, as recommender systems are already quite complex. In a finished product it is expected that all entertainment media forms are included, but for this project, music is going to be excluded to reduce the workload.

It was initially considered that issues like user privacy and rights would be tackled in this project, since a recommendation system uses personal information to generate recommendations. The survey further backed this up, as most people who were asked were not comfortable with their personal information being used. A possible solution was suggested in section \ref{UserRights} on page \pageref{UserRights}, giving the users the possibility to choose whether or not they give consent to using specific personal information, with less accurate recommendations as a consequence should they decline. Despite this the problem with user rights is difficult to answer as it is quite an ethical problem and therefore out of this projects scope. For this reason user privacy and rights will not be looked at further in this project.

The target audience for this project is hard to set, as entertainment media is something almost any age group consumes to some degree. What has been determined is that the project will not cater to children. This is done to prevent making the design process more difficult when filters to protect the children would have to be included. People with physical disabilities or other disadvantageous which prevents normal use of personal computers is also not included in the target audience.

An age group for the target audience can be specified by looking through the survey results. In one of the interviews, a 52-year old interviewee showed no interest in the overall concept of this project and was less aware of recommendation possibilities that can be generated through websites like IMDB. This trend became more apparent depending on how old the interviewee was. The questionnaire showed that all age groups have some form of interest in entertainment media. Considering what has already been stated regarding children the lower bounds of the target age group can be set to 17 years-old. The upper bounds are less clear and can be set anywhere between the thirties and the forties.