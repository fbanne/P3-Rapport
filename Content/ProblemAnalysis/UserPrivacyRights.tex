For a recommendation system it is required to collect data about the intended recipients of the recommendations. The recommendations become better and more precise, as the amount of data becomes bulkier, so a recommendation system will usually try to gather as much it possibly can. This can also be called data profiling, where the intended system for this project would categorize people depending on their taste in media, personal information like age and sex, and their common friend connections. In other instances it could might also include variables like previous purchases, view history, tags, and keywords extracted through text analysis. All of these might be more revealing, and raises privacy concerns.\cite{UserRights2}

Even with this amount of data gathering this project plans to handle, there will be some data privacy concerns. Users will most likely be required to register a profile or account to utilize the system, and will have to hand over personal information for verification and recommendation purposes. This is of course not any different from many other systems who does the same thing. But in this case, we’re talking about a recommendation system, which poses some new challenges in this aspect. Because recommendations can be based on the data of other people, it would make it possible to deduce connections to other people. Especially if it is through some uncommon element, like an obscure cambodian film, where fewer people is connected to. This can ultimately lead to personal information being found out. This risk is further heightened if the person also has access to the database, and can place queries. \cite{UserRights1}

This is a glaring challenge for any recommendation system, but is also a complex and difficult problem to solve, or even answer to. It also depends on exactly how this projects product is going to be constructed, which is not yet formalized, to properly answer these questions. At this point it might be out of this project's scope.

We conducted some interviews where we asked potential users several questions regarding the project, including questions regarding how much information they would be okay with being available, either to a company or the entire public. See Section \ref{Interview} The questions was asked in a rising degree, to see exactly where people’s threshold would be. Most people were okay with their contact information being available, as it was most likely already available in some form of way. When it came to their more personal information, like interests and age, they were more uncertain. Almost all was okay with having their very personal information, like emails, chats, and pictures, available, as expected. An interesting feature a lot of them mentioned were an option to choose whether or these information was shared, for the user themselves. This could be a more user-engaging way to collect information for the recommendation system, and solves most problems regarding user rights, as the user decides for themselves what they want to share.
